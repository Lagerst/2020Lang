%\documentclass{article}
%\usepackage{xeCJK}
\documentclass[UTF8]{ctexart}
\usepackage{fontspec} 
\usepackage{indentfirst} 
\setlength{\parindent}{2em}

\usepackage{enumitem}
\setenumerate[1]{itemsep=0pt,partopsep=0pt,parsep=\parskip,topsep=5pt}
\setitemize[1]{itemsep=0pt,partopsep=0pt,parsep=\parskip,topsep=5pt}
\setdescription{itemsep=0pt,partopsep=0pt,parsep=\parskip,topsep=5pt}

\usepackage{graphicx}
\usepackage{caption}
\usepackage{fontspec, xunicode, xltxtra} 
\usepackage{ctex}
\usepackage{float}

\title{“同心同行\,共克时艰”\,主题班会总结报告}
\author{2017级1102班班委}
\date{}

\begin{document}
	
	\maketitle
	\begin{flushright}
		\date{2020年2月27日}
	\end{flushright}
	\title{\vspace{5em}}
	\begin{abstract}
	本文档为记录2020年2月27日“同心同行\,共克时艰”主题班会的报告。本次班会根据年级要求,由班长李航宇主持对学院政策,疫情防控,学业规划,青年使命等方面进行了建议与总结。
	\end{abstract}
	\thispagestyle{empty}%

	\newpage
	\setcounter{page}{1}
	\section*{概述}
	\title{\vspace{1em}}
	\par{2020年2月27日下午四时,我们班级召开了题为“同心同行\,共克时艰”的主题班会。受疫情影响,本次班会在bilibili上以远程方式召开。本次班会的主要内容包括:}
	\begin{itemize}
	    \item[-]\par{组织学习党中央国务院关于疫情防控的一系列重大决策部署。}
	    \item[-]\par{疫情科普,帮助同学们正确认识疫情,科学了解疫请。}
		\item[-]\par{围绕如何保持良好心态、坚持规律学习、培养和谐家庭关系等话题进行线上讨论。}
		\item[-]\par{开展学业规划讨论。}
		\item[-]\par{围绕青年大学生的责任和使命,讨论如何发挥大学生的作用。}
		\item[-]\par{组织同学们通过多种形式展示学习状况,居家生活,公益举动,祝福感悟等。}
	\end{itemize}
	\title{\vspace{1em}}
	\begin{figure}[H]
		\centering
		\begin{center}{}
			\begin{minipage}[t]{4.5in}
				\includegraphics[width=4.4in]{pre.png}
				\caption*{\kaishu 班会直播前准备}
			\end{minipage}
		\end{center}
	\end{figure}

	\title{\vspace{2em}}
	\section*{学院政策宣读}
	\title{\vspace{1em}}
	\par{在班会的最开始,班会主持人李航宇着重提醒了所有同学实习、选课、缓补考以及重新学习等的通知、政策与时间节点等。关于考研准备,学分清理以及其他更多事项也有提及。同学们积极通过弹幕向主持人提问,主持人耐心解答学院政策,为同学们解答疑惑。主持人还号召同学们参与四川省“青年大学习”主题网上主题团课,发挥青年使命,履行青年责任。}
	
	\title{\vspace{1em}}
	\begin{figure}[H]
		\centering
		\begin{center}{}
			\begin{minipage}[t]{4.5in}
				\includegraphics[width=4.4in]{1.png}
				\caption*{\kaishu 实习政策宣读}
			\end{minipage}
		\end{center}
	\end{figure}
	\title{\vspace{1em}}
	\begin{figure}[H]
		\centering
		\begin{center}{}
			\begin{minipage}[t]{4.5in}
				\includegraphics[width=4.4in]{2.png}
				\caption*{\kaishu 退补选通知宣讲}
			\end{minipage}
		\end{center}
	\end{figure}
	\title{\vspace{1em}}
	\begin{figure}[H]
		\centering
		\begin{center}{}
			\begin{minipage}[t]{4.5in}
				\includegraphics[width=4.4in]{3.png}
				\caption*{\kaishu 缓补考与重新学习通知宣讲}
				\end{minipage}
			\end{center}
		\end{figure}
	\title{\vspace{1em}}
	\begin{figure}[H]
		\centering
		\begin{center}{}
			\begin{minipage}[t]{4.5in}
				\includegraphics[width=4.4in]{4.png}
				\caption*{\kaishu 青年大学习团课}
			\end{minipage}
		\end{center}
	\end{figure}
	
	\title{\vspace{2em}}
	\section*{疫情科普与预防}
	\title{\vspace{1em}}
	\par{之后,主持人对疫情进行了科普,并对同学们如何预防疫情侵害作了一定的建议。主持人建议同学们主动采取预防措施,避免主动或被动暴露在病毒环境中。注意手部卫生,且避免在未洗手的状况下接触口、鼻、眼等部位,出门佩戴符合标准的口罩。避免前往疫区或与患病者密切接触,并保持居住环境的清洁卫生。}	
	\title{\vspace{1em}}
	\begin{figure}[H]
		\centering
		\begin{center}{}
			\begin{minipage}[t]{4.5in}
				\includegraphics[width=4.4in]{5.png}
				\caption*{\kaishu 新型冠状病毒科普}
			\end{minipage}
		\end{center}
	\end{figure}
	
	\title{\vspace{2em}}
	\section*{疫情防控决策部署政策学习}
	\title{\vspace{1em}}
	\par{接着,班会主持人组织学习党中央国务院关于疫情防控的一系列重大决策部署。根据人民网《中央战“疫”日志》相关报道,集中根据时间顺序宣读了党中央国务院1月20日以来的部分重大举措。生命重于泰山,守护人民群众生命安全和身体健康就是守护初心。新冠肺炎疫情发生以来,党中央高度重视,习近平总书记亲自部署、亲自指挥,多次召开会议、多次听取汇报、作出重要指示。与此同时,党中央成立应对疫情工作领导小组,向湖北等疫情严重地区派出指导组,深入一线抗击疫情。通过学习党中央国务院的政策指令,同学们对于抗击疫情的方法举措有了一定的认识,并且坚定地相信伟大的中国人民一定能够彻底击败这次疫情的侵袭。疫情无法摧毁中国人民奋起抗击的坚强意志,相反要被中国人民彻底战胜。}
	\title{\vspace{1em}}
	\begin{figure}[H]
		\centering
		\begin{center}{}
			\begin{minipage}[t]{4.5in}
				\includegraphics[width=4.4in]{6.png}
				\caption*{\kaishu 疫情防控重大决策部署}
			\end{minipage}
		\end{center}
	\end{figure}
	
	\title{\vspace{2em}}
	\section*{保持心态,坚持学习}
	\title{\vspace{1em}}
	\par{主持人再次提示同学们不要提前返校、实习,听从学校安排,充分利用线上课程丰富自己。针对生活与学习,主持人给同学们提出了如下建议:}
	\begin{itemize}
		\item[-]\par{关注自身健康,有必要时及时就诊。我们的身体健康并不是仅仅面对新型冠状病毒的威胁,同样还需要注意其他居家疾病的危害,日常注意消毒、常通风、勤洗手,预防各种季节性传染病的危害,也是同样重要的。}
		\item[-]\par{关注疫情发展,学习防护知识。深居简出,人们就容易与外界环境发生脱节,进而诱发情绪焦虑。因此,正确使用防治知识、积极开展个人防护,并且了解疫情的发展趋势,这对你调整个人心态,避免由于对疫情缺少了解而诱发焦虑非常有帮助。}
		\item[-]\par{注意体育锻炼,关注兴趣爱好。对个人而言,有一个稳定的关注点,才会有比较稳定的情绪状态。在家期间选择慢跑、健身操之类的有氧锻炼,一方面可以确保人体必要的激活水平,另一方面也有助于情绪状态保持稳定。此外,在家隔离,同样可以关注琴棋书画之类的兴趣爱好,这将给闲暇生活带来不少趣味和乐趣。}
		\title{\vspace{1em}}
		\begin{figure}[H]
			\centering
			\begin{center}{}
				\begin{minipage}[t]{4in}
					\includegraphics[width=3.9in]{9.png}
					\caption*{\kaishu 保持心态,坚持学习}
				\end{minipage}
			\end{center}
		\end{figure}
		\item[-]\par{利用互联网、电话等方式与家人、朋友保持沟通。寻求社会支持是人的天性。与家人、朋友保持必要的联系,获得社会支持和心理安慰是维持心理健康的重要手段。虽然隔离期间不宜组织群体聚会,但是通过互联网视频沟通,或者拨打电话进行交流,也是彼此鼓励和抚慰的好方式。}
		\item[-]\par{作息规律、均衡饮食。规律的作息意味着规律的生理节律,意味着较好的精神状态和激活水平。因此,保持规律的作息,摄入均衡的营养,远离酗酒、暴饮暴食等不良的饮食习惯,对保持生理和心理健康同等重要。}
		\title{\vspace{1em}}
		\begin{figure}[H]
			\centering
			\begin{center}{}
				\begin{minipage}[t]{4in}
					\includegraphics[width=3.9in]{8.png}
					\caption*{\kaishu 保持心态,坚持学习}
				\end{minipage}
			\end{center}
		\end{figure}
		\item[-]\par{情绪、心理状态出现波动时,积极进行调整。随着疫情的发展,春季已经开始到来。即便没有疫情,由于季节变化,人们的生理节律也会随之改变。春季和秋季原本就是抑郁症高发的季节,因此,如果出现难以控制的紧张、焦虑或者恐惧,或者出现食欲不振、失眠、体重暴增或骤减,又或者你不愿意与外界沟通,对原本感兴趣的事务失去了兴趣,我们认为这就是出现了明显的心理症状,应该拨打心理咨询热线寻求帮助。}
		\item[-]\par{主持人还举例大学生模范志愿者活动,建议同学们积极参与志愿活动,丰富假期生活。}
	\end{itemize}		
	\title{\vspace{1em}}
	\begin{figure}[H]
		\centering
		\begin{center}{}
			\begin{minipage}[t]{4.5in}
				\includegraphics[width=4.4in]{7.png}
				\caption*{\kaishu 同学们积极进行学业规划讨论}
			\end{minipage}
		\end{center}
	\end{figure}
	\par{同学们积极参与交流,分享自己在假期的学习生活,并表示将规律作息,健康生活,调整心态,一同面对本次新型冠状病毒疫情。留给我们去跨越这场战疫的距离,虽然看起来很遥远,但我们有理由相信,正如阳光暖暖昭示,寒冬已入尾,约定在春暖花开之日再重逢!}

\end{document}